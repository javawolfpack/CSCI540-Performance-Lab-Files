\documentclass[11pt]{article}

\usepackage{times}
\usepackage{epsfig}
\usepackage{url}

%% Page layout
\oddsidemargin 0pt
\evensidemargin 0pt
\textheight 600pt
\textwidth 469pt
\setlength{\parindent}{0em}
\setlength{\parskip}{1ex}


\title{CSCI 540 \& CSCI640 - Systems Programming\\
Assignment 4\\
Performance Lab: Code Optimization\\
}
\author{}
\date{}
\begin{document}
\maketitle

\section{Introduction}
This assignment deals with optimizing memory intensive code.  Image
processing offers many examples of functions that can benefit from
optimization.  In this lab, you'll be improving the overall
performance of an ``image processing'' application by a factor of
about 25 -- if you can increase the speed by a factor of $\approx$ 50,
you'll get extra credit.

The application you'll be modifying reads in an ``image'' (a picture)
and a ``filter''.  An image is represented as a three-dimensional
array, described in \verb|cs1300bmp.h|.  Each pixel is represented as
a combination of (red, green, blue) values -- when coded in the BMP
picture format, the individual (R,G,B) values can taken on the values
$0 \ldots 255$, but the \verb|cs1300bmp.h| code is designed to handle
larger pixel values.  The code in \verb|cs1300bmp.cpp| provides
routines for reading and writing images in the BMP format. You are
free to modify the format or the layout of the data structures in that
code.

The ``filter'' is an $n\times n$ array of numbers.  We'll go through
the logistics of how a ``filter'' works in recitation and in class and
briefly summarize it here. Basically, you an cause a number of visual
affects by applying a filter to an image. The filter is implemented as
a ``convolution'', which means that elements of the filter matrix are
multiplied by the image matrix to compute a new value for the
image. Pictorially, this is represented as:

\begin{figure}[h]
\centering
\includegraphics[width=2in]{writeup-picture}
\caption{Filter Operation}
\label{fig:ops}
\end{figure}

Computationally, this is structured as five nested \verb|for| loops
(three to go over the colors, row and columns and two more to apply
the filter). In the solution provided to you, filters are
represented using the \verb|Filter| class, implemented in \verb|Filter.h|
and \verb|Filter.cpp|.

The majority of the work performed by the filter application is in
the routine shown in Figure~\ref{fig:filter}.

\begin{figure}[h]
\footnotesize
\begin{verbatim}
  long long cycStart, cycStop;
  rdtscll(cycStart);

  output -> width = input -> width;
  output -> height = input -> height;


  for(int col = 1; col < (input -> width) - 1; col = col + 1) {
    for(int row = 1; row < (input -> height) - 1 ; row = row + 1) {
      for(int plane = 0; plane < 3; plane++) {
        output -> color[plane][row][col] = 0;
        for (int j = 0; j < filter -> getSize(); j++) {
          for (int i = 0; i < filter -> getSize(); i++) {
           output -> color[plane][row][col] = output -> color[plane][row][col] 
            +  input -> color[plane][row + i - 1][col + j - 1] * filter -> get(i, j);
          }
        }
        output -> color[plane][row][col] 
         = output -> color[plane][row][col] / filter -> getDivisor();
        if ( output -> color[plane][row][col]  < 0 ) { output -> color[plane][row][col] = 0; }
        if ( output -> color[plane][row][col]  > 255 ) { output -> color[plane][row][col] = 255; }
        output -> color[plane][row][col] = output -> color[plane][row][col];
      }
    }
  }
  rdtscll(cycStop);
  double diff = cycStop - cycStart;
  fprintf(stderr, "Took %f cycles to process, or %f cycles per pixel\n",
          diff, diff / (output -> width * output -> height));
\end{verbatim}
\caption{Core of filter code}
\label{fig:filter}
\end{figure}

This routine is ``instrumented'' using the \verb|rdscll|
function. This inline function records the
starting and finish times in terms of CPU cycles. We use this
to determine the ``cycles per element'' needed to apply the
filter to a given image. A sample of the output for the 
provided setup code looks like the following when run on
your machine.

\begin{verbatim}
Took 170624720.000000 cycles to process, or 3229.816007 cycles per pixel
\end{verbatim}

The cycle counter measures times using the CPU clock of your computer.
It is fairly accurate, but many things (such as other running
programs) influence the reported time. Thus, we will use the {\em
  median} of a number of runs to determine the time for a particular
implementation of the program for a number of different filters.

Your job is going to be to improve the performance of this application
using techniques detailed in Chapter 5 of the text. You'll be using
two images (\verb|boats.bmp| and \verb|blocks-small.bmp|) The first
image is fairly small (useful for quickly testing ideas) and the
second image is larger (and used for grading / evaluation). You can
run your program using a command line similar to this:

\begin{verbatim}
$ ./filter hline.filter boats.bmp
\end{verbatim}

This invocation will leave the output image in \verb|filter-hline-boats.bmp|
and will report the time taken, as in the examples above. You can repeat
the image name (or use different images) on the same command line to run
the filter multiple times \& return the average. For example:

\begin{verbatim}
$ ./filter hline.filter boats.bmp boats.bmp 
Took 139255936.000000 cycles to process, or 2636.025138 cycles per pixel
Took 137527184.000000 cycles to process, or 2603.300977 cycles per pixel
Average cycles per sample is 2619.663057
\end{verbatim}

We'll be using a script called \verb|Judge| to score your program. The
\verb|Judge| program takes three optional argument. The \verb|-n|
argument specifies the number of times each filter should be executed,
the \verb|-i| specifies the image file and \verb|-p| specifies
the program name. The default options are shown explicitly in the
following example.

\begin{verbatim}
% ./Judge -p filter -n 6 -i blocks-small.bmp
gauss: 2852.515711..2814.703339..2791.057323..2808.909958..2809.439831..2867.625660..
avg: 2781.929192..2843.245386..2825.327816..2807.778870..2862.825325..2836.457027..
hline: 2923.815990..2881.138340..2889.208279..2856.241310..2899.939484..2866.064194..
emboss: 2843.983543..2873.607147..2819.075932..2836.724133..2891.809189..2851.489498..
Scores are 2781 2791 2807 2808 2809 2814 2819 2825 2836 2836 2843 2843 2851 2852 2856 2862 2866 2867 2873 2881 2889 2891 2899 2923 
median CPE is 2851
Resulting score is 37
\end{verbatim}

This would result in an average of 2851 cycles per second. Scoring is based
on the CPE results for the \verb|blocks-small.bmp| image
running on the \verb|perf| machines. The \verb|perf| machines are Freescale i.MX 6 Quad Core ARM processors, you can run your code on the \verb|perf| machines through the web interface at \url{http://bryancdixon.com:8002}.


The  provided Makefile compiles your program; you may need to modify it to
change compiler options or the like. You can also compile your program
"by hand", but you should remember what you did. We assume you are compiling
your program on your virtual machine. The Makefile also provides a \verb|make judge| rule that runs the test
images and filters. Lastly, it provides a \verb|make clean| rule to delete any temporary files or images. I would recommend commenting out the judge rule if you've modified the Makefile and want to use the new Makefile to run your code as the \verb|make judge| rule will run as part of the compilation process. 

Your measured time needs to include any active processing you do to
the image after it is read in and before it is written out.  You're
free to go ``whole hog'' on any optimization that might work, as long
{\bf as it works with all the test images and filters included in the
  assignment}. Your modified code {\bf does not} need to handle any
filters or images not included in the evaluation suite.  This means you
can go ahead and {\em e.g} change the matrix layout in the BMP image
library, replace the Filter library, {\em etc.} However, you'd be well
advised to make certain those changes are important and effective
before you sink a lot of time.

The most ``extreme'' solution is to use optimized functions for every filter. 
The \verb|perf| machines also four CPU's or cores and you could use the OpenMP language
extensions to try to use all the cores. Some students have used these methods in
the past, but you're well advised to go for the easy low-hanging
fruit before tackling the most aggressive optimization.

\section{Logistics}
�The only ``hand-in'' will be electronic.
Any clarifications and revisions to the assignment will be posted on
the Piazza discussion board.

Different computers will run this code at different speeds (even
measured in ``cycles per second''). In order to have a level
playing field, everyone must use the \verb|perf| machines to time their projects. The web portal interface will insure that you are the only person running your code on the \verb|perf| machine at the time insuring a consistent result. 

You should do your development on your virtual machine. Since it's
usually true that making something run faster on one machine makes it
run faster on another, you might be able to do your development on one
machine and then use the ``perf'' machines simply to validate your
measurements or improvements.

Download the \verb|perflab-setup.zip| file right away and unzip it.
Make certain you can compile and run the distributed version
and achieve times comparable to those above. You should also create a second copy of the original files so you can filter the pictures and check that they are identical to the filtered results your solution generates as they have to be {\bf identical} for you to get any credit for this assignment. You should also make use of \verb|git| version control so you can roll back should the photos stop filtering correctly and so you can commit every change and your justification for why it makes your code run faster, which'll be helpful for the grading quiz. 

\section{Grading} 

You should upload your files to Turnin by the due date.

You need to be able to explain {\em why} your code modifications
improve the running time for the program and explain what would
happen with minor modifications.

Again, you must be able to explain {\em why} you got the performance
you did, so you should take notes for why you made each modification
to bring to the lecture with the grading quiz for this assignment.

\end{document}

